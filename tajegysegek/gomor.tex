\chapter{Gömör}
	\section{Van két lovam...}
	Van két lovam, van két lovam, mind a kettő sárga,\\
	Még az éjjel átugratok Csehszlovákiába,\\
	Még onnan is visszanézek szép Magyarországra,\\
	Fáj a szívem édesanyám, meghalok utána.\\\\
	Megtiltották a magyar szót Csehszlovákiában,\\
	Felvidéki magyar gulyás gondolja magában,\\
	Bíró uram új gulyásról tessék gondoskodni,\\
	Nem tudok én a birkára tótul káromkodni.\\\\
	Kimegyek a temetőbe, oszt’ beszélek a csősszel,\\
	Ássa meg az én síromat még az idén ősszel,\\
	Egy sír helyett kettőt ásson, az kevés lesz nékem,\\
	Az egyiket a bánatomnak, a másikat nékem.\\
	\section{Az aradi fegyház...}
	Az aradi fegyház kőből van kirakva,\\
	az alatt sírdogál egy jó édesanya.\\
	Ne sírj édesanya, így kell annak lenni,\\
	minden jó családból kell egy rossznak lenni.\\\\
	Amerre én járok, még a fák is sírnak,\\
	gyenge ágaikról levelek lehullnak.\\
	Hulljatok levelek, rejtsetek el engem,\\
	mert akit szeretek mást szeret, nem engem.\\
	\section{Elveszett a lovam...}
	\lilypondfile[staffsize=16]{nwc_files/gomor/gomor_elveszett_a_lovam.lytex}\\
		Elveszett a lovam,\\
		cédrusfa erdőbe’.\\
		Elkopott a keményszárú csizmám,\\
		a lókeresésbe’.\\\\
		Ne keresd a lovad,\\
		mert be van az hajtva.\\
		A nagybalogi bíró udvarába’\\
		szól a csengő rajta.\\\\
		Ismerem a lovam,\\
		csengő szólásáról.\\
		Ismerem a kedves kisangyalom\\
		a büszke járásáról.\\
	\section{Sok irigyem a faluba'...}
		\lilypondfile[staffsize=22]{nwc_files/gomor/gomor_sok_irigyem_a_faluba.lytex}
		Sok irigyem a faluba’,\\
		kettő-három egy kapuba’.\\
		El akarnak veszejteni,\\
		jó az Isten, nem engedi.\\\\
		Sokszor kértelek a jóra,\\
		ne menj a kocsmaajtóra.\\
		Mert betörik a fejedet,\\
		kutyák isszák a véredet!\\\\
		Édesanyám sok szép szava,\\
		kire nem hajlottam soha.\\
		Hallgatnék már, de már késő,\\
		hull a könnyem, mint az eső.\\