\chapter{Szék}
\section{A kapuba' a szekér...}
\lilypondfile[staffsize=14]{nwc_files/szek/szek_a_kapuba_a_szeker.lytex}
A kapuba’ a szekér,\\
Itt a legény, leányt kér,\\
De a leány azt mondja,\\
Nem megyek férjhez soha.\\\\
Örömanya jöjjön ki,\\
Hogy az ajtót nyissa ki,\\
Eressze be a vejét,\\
A lánya szeretőjét!\\
\section{Kimentem én az erdőbe, az erdőbe...}
Kimentem én az erdőbe, az erdőbe,\\
Ráléptem egy venyigére, venyigére,\\
Venyigéről venyigére, venyigére,\\
Fáj a szı́vem a szőkére, a szőkére.\\\\
Kimentem én a piacra, a piacra,\\
Ráléptem egy papirosra, papirosra,\\
Papirosról papirosra, papirosra,\\
Fáj a szı́vem a pirosra, a pirosra.\\\\
Kimentem én az utcára, az utcára,\\
Ráléptem egy szalmaszálra, szalmaszálra,\\
Szalmaszálról szalmaszálra, szalmaszálra,\\
Fáj a szı́vem a barnára, a barnára.\\
\section{Aj Istenem, adj egyet...}
Aj Istenem, adj egyet,\\
Soha nem kérek többet!\\
Aj Istenem, adj egy jót,\\
Adj egy kedvemre valót!\\\\
Aj Istenem, adj egy krajcárt,\\
Hadd vegyek egy kis pálinkát!\\
Aj Istenem, adjál kettőt,\\
Hadd vegyek egy szép szeretőt!\\\\
Azt hallottam én a héten,\\
Leányvásár lesz a réten,\\
Én is oda fogok menni,\\
Szőkét, barnát választani.\\\\
Szőkét ne végy, mert beteges,\\
Pirosat se, mert részeges,\\
Barnát vegyél, az lesz a jó,\\
Az lesz az ölelni való.\\
\section{Bonchidai menyecskék...}
Bonchidai menyecskék\\
Ugrálnak, mint a kecskék,\\
Szeretőjük mindig más,\\
Az uruk csak ráadás.\\\\
Bonchidai hı́d alatt\\
Lányok sütik a halat.\\
Papirosba csavarják,\\
A legénynek úgy adják.\\\\
Bonchidai hı́d alatt\\
Leány a legény alatt.\\
Azért feküdt alája,\\
Viszketett a szoknyája.\\
\section{Tele van a temető árka vízel...}
Tele van a temető árka vízzel,\\
Tele van a szívem keserűséggel.\\
Jó az Isten, elapassza a vizet,\\
Sej, visszahozza a régi szeretőmet.\\\\
Hideg sincsen, mégis befagyott a tó,\\
Ihatnék az én szép lovam, a fakó.\\
Gyertek lányok, törjétek be a jeget,\\
Sej, hadd igyon a fakó lovam eleget!\\\\
Nem szeretem az uramat, nem biz’ én,\\
Ha kimegyek, ha bemegyek, sírok én.\\
Sírok, sírok, siratom a lányságom,\\
Siratom a lánykori boldogságom.\\
\section{Erdő, erdő, kerek erdő...}
Erdő, erdő, kerek erdő,\\
De szép madár lakja kettő,\\
Kék a lába, zöld a szárnya,\\
Piros a rózsám orcája.\\\\
Olyan piros, mint a vér,\\
Tőlem gyakran csókot kér\\
De én bizony nem adok,\\
Inkább jól megátkozom.\\\\
Kilenc fia néma legyen,\\
A tizedik leány legyen,\\
Az is olyan csalfa legyen,\\
Ország-világ híre legyen.\\\\
Kinek nincsen szeretője,\\
Menjen ki a zöld erdőbe,\\
Írja fel egy falevélre,\\
Neki nincsen szeretője.\\\\
Kinek nem jó itt lenn lakni / Kinek nincs kedve itt lakni,\\
Menjen mennyországban lakni,\\
Építsen az égre házat,\\
Ott nem éri semmi bánat.\\\\
Építsen az ég szélére,\\
Ott nem éri semmiféle,\\
Építsen az égre házat,\\
Ott nem éri semmi bánat.\\\\
Verd meg Isten azt a helyet,\\
Ahol az én rózsám termett,\\
Hogy ne teremjen több rózsát,\\
Bús szívem szomorítóját.\\\\
Olyan furcsa kedvem vagyon,\\
A szõkét szeretem nagyon,\\
A barna sem ellenségem,\\
Most is õ a feleségem.\\\\
Szeress, szeresd, de nézd meg, kit,\\
Mer’ a szerelem megvakít,\\
Engemet is megvakított,\\
Örökre megszomorított.\\

Fonó Zenekar: Hateha! 1. Széki - 6:11-től:\\
https://www.youtube.com/watch?v=0S5gDaF\_chI\&t=6m11s

Téglás Zenekar: Sűrű, ritka tempó és magyar - Táncház - Népzene 2001 9. - 02:37-től

\section{Édesanyám, miért is szültél...}
Édesanyám mért is szültél,\\
ha engemet megvetettél?\\\\
Mért is szültél e világra,\\
örökös szomorúságra.\\
Édesanyám is volt nékem,\\
nem is olyan réges-régen.\\\\
Nem is olyan rég nincs anyám,\\
csak mióta meghalt apám.\\
Édesanyám, hol a szíved,\\
Hol az anyai szerelmed?\\\\
Mért virágzik jégvirágot,\\
mért vetetted meg a lányod?\\
Kimegyek a temetőbe,\\
leborulok egy sírkőre.\\\\
Az a sírkő kinek jele,\\
ott van apám eltemetve.\\
Leborulok sírhalmára,\\
szívem vigasztalására.\\\\
Göröngyei azt susogják,\\
itt vár reám a boldogság.\\
Itt lesz szíved meggyógyulva,\\
itt lesz lelked megnyugodva.\\\\
Édesanyám miért is szültél,\\
ha ennyire megvetettél?\\
De sok mindennek kitettél,\\
a bánattól nem kíméltél.\\

Herczku Ágnes: Arany és kék szavakkal 5. Egy menyasszonyé - 4:14-től:
http://www.youtube.com/watch?v=JZGE1r8mbb8\}\&t=4m14s

Téka Együttes: Ha te húzod, én meg járom 1. Széki magyar és csárdás:
https://www.youtube.com/watch?v=sAy8vh5x6I8

Angyalföldi Vadrózsa Táncegyüttes: Szék városa - 1:32-től és 16:16-tól:
https://www.youtube.com/watch?v=Fn1j91x0pvU\&t=1m32s

\section{Szennyes ingem, szennyes gatyám...}
Szennyes ingem, szennyes gatyám,\\
Mezőségen lakik anyám,\\
Nincsen fája, sem hamuja,\\
Hogy az ingem megszapulja.\\
Ne nézd, hogy én szennyes vagyok,\\
Mert szívemben nincsen mocsok,\\
Az az egy pecsét benne van,\\
Szeretlek én, rózsám nagyon.\\\\
Itt is terem sok szép virág,\\
Madár dalol, szép a világ,\\
De az mind semmit ér nekem,\\
Ha a babám nem ölelem.\\
Es' az eső szép csendesen,\\
Ne sírj rózsám keservesen,\\
Ne sírj rózsám keservesen,\\
Majd megsegít a jó Isten.\\

Poros zenekar + Százszorszép táncegyüttes - 01:31-től:
http://www.youtube.com/watch?v=ui4G1\_diJpA\&t=1m31s

Fonó Zenekar: Hateha! 1. Széki - 1:44-től:
https://www.youtube.com/watch?v=0S5gDaF\_chI\&t=1m44s

Kovács Nóri: Azt hallottam, jó galambom... - Táncház - Népzene 2001 8. - 01:27-től

\section{Lefelé folyik a Tisza...}
Lefelé folyik a Tisza,\\
Nem folyik az soha vissza,\\
Rajtam van a babám csókja,\\
a sajnálja, vegye vissza.\\\\
Ha sajnálja, miért is adta,\\
Nem vagyok én rászorulva,\\
Ha sajnálja, miért is adta,\\
Nem vagyok én rászorulva.\\\\
Édesanyám mondta nékem,\\
Minek a szerető nékem,\\
De én arra nem hallgattam,\\
Titkon szeretőt tartottam.\\\\
Mindenkinek azt ajánlom,\\
Szerelemnél jobb az álom,\\
Mer' az álom nyugodalom,\\
A szerelem szívfájdalom.\\\\
Szeress rózsám, csak nézd meg kit,\\
Mer' s a szerelem megvakít,\\
Engemet is megvakított,\\
Örökre megszomorított.\\

Poros zenekar + Százszorszép táncegyüttes - 03:15-től:
http://www.youtube.com/watch?v=ui4G1\_diJpA\&t=3m15sH



