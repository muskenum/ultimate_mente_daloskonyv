\chapter{Vág-Garam köze}
\section{Hosszú farkú fecske, barna kismenyecske...}
Hosszú farkú fecske, barna kismenyecske,\\
Hogy tudtál te ide jönni, ez idegen földre?\\\\
Nem jöttem én ide, kocsin hoztak ide,\\
Az én babám csalfa szeme csalogatott ide.\\\\
\textit{Kéménd}

\section{Már minálunk a barátok facipőbe' járnak...}
\lilypondfile[staffsize=14]{nwc_files/vgk_mar_minalunk_a_baratok_facipobe_jarnak.lytex}
Messze mentem házasodni, bár në mentëm volna,\\
Édes kedves felesigëm, csak ott vesztél volna,\\
Elvëttem én egy vén leánt, meg is estëm véle,\\
Csiszeg-csoszog a papuccsal, jaj dë félëk tőle.\\\\
Vajon esküdt, vajon bíró válasszon el tőle,\\
Isten uccse, teremt uccse, meg nem élek véle.\\
Ha az ördög olyan volna, taligára tënné,\\
Mennél jobban sikojtana, annál jobban vinné.\\\\
\textit{Martos}

\section{Nem forog a nem forog a dorozsmai szélmalom...}
Nem forog a nem forog a dorozsmai szélmalom,\\
Levelet írt, levelet írt az én kedves galambom,\\
Mit ér nékem a százaranyos levele,\\
Ha én magam, ha én magam nem beszélhetek vele.\\\\
\textit{Tardoskedd, 1966}

\section{Felszállott a kakas a létrára...}
Felszállott a kakas a létrára,\\
Kukorékol hajnal hasadtára,\\
Hajnal hasad, csak egy csillag ragyog\\
Én meg babám most is nálad vagyok.\\\\
Látod babám ezt a tearózsát,\\
nappal nyílik, este járok hozzád,\\
este járok, mikor senki se lát,\\
mégis irigy rám az egész világ.\\\\
\textit{Kéménd}

\section{Áll a malom, áll a vitorlája...}
Áll a malom, áll a vitorlája\\
Barna kislány, mi lesz vacsorára\\
Halpaprikás, mi gondja van rája\\
Estére várom a babám vacsorára.\\\\
Már minálunk az jött a szokásba,\\
Nem táncolnak alacsony szobába,\\
Alacsony a szoba mestergerendája,\\
Sej betörik a babám bokrétája.\\\\
\textit{Kéménd}

\section{Temető kapuja, mind a kettő nyitva...}
Temető kapuja, mind a kettő nyitva\\
Bárcsak engem temetnének oda\\
Hej, abba a fekete sírba.\\\\
A sírom teteje, rózsával lesz tele,\\
Az izsai híres szobalányok,\\
Sej, szedik rólam a virágot.\\\\
Szedjétek csak lányok, rólam a virágot\\
Csak azt a szép fehér liliomot,\\
Sej, rólam el ne raboljátok.\\\\
\textit{Izsa, 1968}

\section{Naszvad község szép helyen van...}
Naszvad község szép helyen van,\\
Közepében szép templom van,\\
Körös körül rózsa csipke\\
Rászállott a bús gerlice.\\\\
Hogyha bús gerlice volnék,\\
Babám ablakjára szállnék,\\
Mindig csak azt turbékolnám,\\
Ébren vagy-e kedves rózsám.\\\\
Ébren vagyok, nem aluszok,\\
Mindig rólad álmodozom,\\
Így jár aki szeretőt tart,\\
Éjjel nappal nem alszik az.\\\\
\textit{Naszvad, 1972}

\section{Kimentem a rétre, nem tudtam kaszálni...}
Kimentem a rétre, nem tudtam kaszálni,\\
Arra jött a rózsám, meg akart ölelni,\\
Tizenhárom kasza mind egyszerre vágjo,\\
Tizenhárom csókot kapsz a pici szádro,\\
Ibolya, viola, gyönge szellő fújja,\\
Ibolya levélről leszakadt a rózsa.\\\\
Elvágtam az ujjam, folyik piros vérem,\\
Nincsen nekem anyám, ki bekösse nékem,\\
Gyere kisangyalom, ne röstöld bekötni,\\
Ha én nem röstölök hozzád későn járni,\\
Ibolya, viola, gyönge szellő fújja,\\
Ibolya levélről leszakadt a rózsa.\\\\
\textit{Martos, 1957}

\section{Sárga paszuly az ágy alatt...}
\lilypondfile[staffsize=14]{nwc_files/vag_garam_koze/vgk_sarga_paszuly_az_agy_alatt.lytex}\\\\
Fecském, fecském, édes fecském,\\
Videe el az én levelecském.\\
Vidd ell, vidd el, tyuhaj, messze tájra,\\
Tedd a babám ablakába.\\\\
Hogyha kérdi, honnan jöttél,\\
Kinek a póstása lettél,\\
Mondd meg, mondd meg neki, hogy egy lánynak,\\
Az ő régi babájának.\\\\
\textit{Tóbiás András (1913), Mokcsakerész (Krížany), 1958\\ Gyűjtötte: Gyüre Lajos}
